\section*{Hybrid Search Results}
\begin{itemize}
  \item \textbf{File:} Eugene Hecht - Optics-Pearson (2016).pdf

  \texttt{234	 Chapter 5  Geometrical Optics Because many objects of interest to astronomers—planets,  galaxies, nebulae, and so on—are imaged as extended bodies,  using these as an adaptive-optics beacon is pr}

  \item \textbf{File:} Eugene Hecht - Optics-Pearson (2016).pdf

  \texttt{a set of apertures spaced at about h or less should produce nice  fringes. Modern Astronomical Interferometry Today Michelson’s stellar interferometer has morphed into a  variety of magnificent ultra-}

  \item \textbf{File:} Eugene Hecht - Optics-Pearson (2016).pdf

  \texttt{significance, and the diffraction-limited system with an appre- ciable field of view became a reality. The technique of ion bom- bardment polishing, in which one atom at a time is chipped  away, was i}

  \item \textbf{File:} Eugene Hecht - Optics-Pearson (2016).pdf

  \texttt{230	 Chapter 5  Geometrical Optics based optical telescope arrays are destined to contribute signifi- cantly to the way we see the Universe. Catadioptric Telescopes A combination of reflecting (catopt}

  \item \textbf{File:} Eugene Hecht - Optics-Pearson (2016).pdf

  \texttt{*See L. A. Thompson, “Adaptive Optics in Astronomy,” Phys. Today 47, 24 (1994);  J. W. Hardy, “Adaptive Optics,” Sci. Am. 60 (June 1994); R. Q. Fugate and W. J.  Wild, “Untwinkling the Stars—Part I,” }

  \item \textbf{File:} Eugene Hecht - Optics-Pearson (2016).pdf

  \texttt{case in astronomy and spectroscopy. For example, the star Sirius,  which appears as the brightest star in the sky (it’s in the constel- lation Canis Major—the big dog), is actually one of a binary  sy}

  \item \textbf{File:} Eugene Hecht - Optics-Pearson (2016).pdf

  \texttt{cato of impacts that are random in space and time across the  beam. Suppose that we project a light pattern onto the screen;  it might be a set of interference fringes or the image of a  TABLE 3.1    }

  \item \textbf{File:} Dietrich Korsch (Auth.) - Reflective Optics-Academic Press (1991).pdf

  \texttt{forming properties of reflecting conic sections of revolution.  2.1 FERMAT'S PRINCIPLE  The theory of geometrical optics can be developed on the basis of a  single hypothesis, known as Fermat's princi}

  \item \textbf{File:} Dietrich Korsch (Auth.) - Reflective Optics-Academic Press (1991).pdf

  \texttt{field of reflective optics and its importance have increased enor­ mously. Therefore, but also because of the fundamental differences  that indeed exist between refractive and reflective optics, it is}

  \item \textbf{File:} Eugene Hecht - Optics-Pearson (2016).pdf

  \texttt{of the key points in what is to follow is that optical elements  frequency. Accordingly, it would seem reasonable to design the  optics preceding such detectors so that it provided the most con- trast}

\end{itemize}
